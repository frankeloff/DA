\documentclass[12pt]{article}
\usepackage{fullpage}
\usepackage{multicol,multirow}
\usepackage{tabularx}
\usepackage{ulem}
\usepackage[utf8]{inputenc}
\usepackage[russian]{babel}
\usepackage{pgfplots}

\begin{document}

    \section*{Лабораторная работа №\,9 по курсу дискрeтного анализа: 
    Графы}

    Выполнил студент группы М8О-308Б-20 МАИ \textit{Зинин Владислав}.

    \subsection*{Условие}
 
    \begin{enumerate}
    \item Разработать программу на языке C или C++, реализующую указанный
    алгоритм. Формат входных и выходных данных описан в варианте задания.
    Первый тест в проверяющей системе совпадает с примером.
    \item \textbf{Вариант 3: Поиск компонент связности} Задан неориентированный граф, состоящий из n вершин и m ребер.
    Вершины пронумерованы целыми числами от 1 до n. Необходимо вывести все компоненты связности данного графа.
    \end{enumerate}

    \subsection*{Метод решения}

    Поскольку заданный граф не является ориентированным, можно использовать обход в глубину
    для решения данной задачи. При запуске обхода из вершины, принадлежащей к 
    некоторой компоненте связность, обход посетит все вершины из этой компоненты и только их.
    Таким образом, в функцию обхода можно передать вектор, в который будут помещаться вершины из очередной компоненты связности.
    Считываем заданный неориентированный граф, запускаем обходы в глубину из всех его вершин, сохраняя при этом результат в вектор, передаваемый по ссылке.
    По итогу наша сложность совпадает со сложность обхода в глубину, то есть O(V + E), где V и E = обзее количество верщин и ребер 
    в графе соответственно.

    \subsection*{Описание программы}

    Программа состоит из одного файла.

    \subsection*{Дневник отладки}

    \begin{enumerate}
    \item Программа была выполнена с первого раза.
    \end{enumerate}

    \subsection*{Тест производительности}

    Ниже приведен тест времени работы алгоритма. По оси $X$ — количество 
    вершин в графе, по оси $Y$ — время выполнения алгоритма в мс (меньше 
    — лучше).
    
    \begin{tikzpicture}
        \begin{axis} [
            ymin = 0
        ]
        \addplot coordinates {
            (1000,510) (10000,4980) (50000,32914)
        };
        \end{axis}
    \end{tikzpicture}

    \begin{tabular}{ | l | l | l | }
        \hline
            Кол-во вершин    & Время (в мс) \\ \hline
            1000             & 510           \\
            10000            & 4980          \\
            50000           & 32914         \\
        \hline
    \end{tabular}

    Таким образом, мы видим, что сложность данного алгоритма - линейная.

    \subsection*{Выводы}

    В результате проведенной лабораторной работы я повторил написание обхода неориентированного графа в глубину и 
    научился его модифицировать для поиска компонент связности.

\end{document}