\documentclass[12pt]{article}

\usepackage{fullpage}
\usepackage{multicol,multirow}
\usepackage{tabularx}
\usepackage{ulem}
\usepackage[utf8]{inputenc}
\usepackage[russian]{babel}
\usepackage{pgfplots}
% Оригиналный шаблон: http://k806.ru/dalabs/da-report-template-2012.tex
\usepackage{listings} %% собственно, это и есть пакет listings

\lstset{ %
language=C++,                 % выбор языка для подсветки (здесь это С++)
}
\begin{document}

\section*{Лабораторная работа №\,4 по курсу дискрeтного анализа: Поиск образца в строке}

Выполнил студент группы 08-208 МАИ \textit{Зинин Владислав}.

\subsection*{Условие}

Кратко описывается задача: 
\begin{enumerate}
\item Необходимо реализовать поиск одного образца в тексте с использованием алгоритма Z-блоков. Алфавит — строчные латинские буквы.
\end{enumerate}

\subsection*{Метод решения}

Решение представляет собой реализацию эффективной Z-функции, которая имеет О(n), поскольку каждое значение проходится не более двух раз. Данный алгоритм называется эффективным,
потому что он значительно быстрее наивного алгоритма, который заключается в обычном подсчете схожих элементов и имеет сложность O(n*n). Эффективный способ основан на использовании
информации об уже посчитанных значениях Z-функции для предыдущих значений, а именно использование "границ" и поиск аналогов элементов, находящихся в этих границах. В программе представлены функции
ZFunction, представляющая собой эффективный алгоритм Z-функции,  NaiveZF соответственно, наивный и Result для получения результата, оперируя информацией, полученной с помощью Z-функции. Мы ищем в 
массиве, начиная с позиции текста, значения, равные длине нашего паттерна и выводим их вхождение.

\subsection*{Описание программы}

Программа написана в 1 файле: main.cpp.

\newline std::vector<int> ZFunction (std::string& zF) - эффективный алгоритм Z-функции
\newline std::vector<int> NaiveZF(std::string& text) - наивный алгоритм Z-функции
\newline void Result(std::string& text, std::string&pattern) - функция для вывода результата

\subsection*{Исходный код}

\subsection*{main.cpp}:

\begin{lstlisting}
    #include <iostream>
    #include <vector>
    
    std::vector<int> ZFunction (std::string& zF){
        int n = zF.size();
        int left = 0;
        int right = 0;
        std::vector<int>Res(n);
        Res[0] = zF.size();
        for(int i = 1; i < n; i++){
            if(i > right){
                int j = 0;
                while(zF[i + j] == zF[j] && j < n){
                    ++j;
                }
                Res[i] = j;
                if(i + j > right){
                    right = i + j - 1;
                    left = i;
                }
            }
            else{
                if(Res[i - left] + i - 1 < right){
                    Res[i] = Res[i - left];
                }
                else{
                    int j = right - i + 1;
                    while(zF[j] == zF[j + i] && j + i < n){
                        j++;
                        Res[i]++;
                    }
                    Res[i] += right - i + 1;
                    if(Res[i] + i - 1 > right){
                        right = Res[i] + i - 1;
                        left = i;
                    }
                }
            }
        }
        return Res;
    }
    
    std::vector<int> NaiveZF(std::string& text){
        int n = text.size();
        std::vector<int> Res(n);
        Res[0] = n;
        for(int i = 1; i < n; i++){
            int j = 0;
            while(text[0 + j] == text[i + j] && i + j < n){
                ++j;
            }
            Res[i] = j;
        }
        return Res;
    }
    
    void Result(std::string& text, std::string&pattern){
        std::string zF = pattern + "$" + text;
        std::vector<int> Res = ZFunction(zF);
        for(int i = 0; i < text.size(); i++){
            if(Res[pattern.size() + 1 + i] == pattern.size()){
                std::cout << i << "\n";
            }
        }
    }
    
    int main(){
        std::cin.tie(nullptr);
        std::cout.tie(nullptr);
        std::ios_base::sync_with_stdio(false);
        std::string text;
        std::string pattern;
        std::cin >> text >> pattern;
        std::string zF = pattern + "$" + text;
        Result(text, pattern);
    }

\end{lstlisting}  


\subsection*{Дневник отладки}

Возникли трудности с написанием эффективной Z-функции для случая, когда длина оставшейся строки, схожей с той, что находится в границах, меньше либо равна Z-функции,
равной в данном аналоге (случай Res[i - left] + i - 1 >= right). После продолжительных раздумий я смог определить, что позиция следующего элемента за данным определяется по формуле right - i + 1,
что так же является значением Z-функции в позиции, являющейся аналогом искомой.


\subsection*{Тест производительности}

Для теста производительности я написал генератор, который генерировал текст + паттерн, текст и паттерн состояли из одинаковых букв. Всего тестов - 10000.
Паттерн меньше текста в 10 раз. Каждый новый тест я увеличивал длину текста и паттерна, результаты получились следующими:
\newline (Naive || Z-func)
\newline MAX LENGTH = 100 - 190 ms || 75 ms
\newline MAX LENGTH = 200 - 585 ms || 190 ms
\newline MAX LENGTH = 400 - 1945 ms || 300 ms
\newline MAX LENGTH = 800 - 7124 ms || 520 ms
\newline MAX LENGTH = 1600 - 27456 ms || 964 ms


\begin{tikzpicture}
    \begin{axis}[
        xlabel = {$x$},
        ylabel = {$y$},
        legend pos = north west,
    ]
    \legend{ 
        $\Naive$, 
        $\ZFunc$
    };
    \addplot coordinates {
        (100,190) (200,585) (400,1945) (800,7124) (1600,27456)
    };
    \addplot coordinates {
        (100,75) (200,190) (400,300) (800,520) (1600,964)
    };
    \end{axis}
\end{tikzpicture}

Синий - наивный алгоритм, красный - эффективный.


\subsection*{Недочёты}

Недочетов не обнаружено.

\subsection*{Выводы}

В данной лабораторной мною была реализована программа, производящая поиск паттерна в строке. Я познакомился с наивным и эффективным алгоритмами построения Z-функции и использованием
её для нахождения паттерна в тексте. Также я сравнил эффективный алгоритм с наивным и убедился в том, что эффективный алгоритм работает за линейное время.

\end{document}